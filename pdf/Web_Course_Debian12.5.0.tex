% TeXShop要选择XeLaTeX才能编译中文{ctexart}
\documentclass[UTF8,a4paper,12pt]{ctexbook} %中文支持,书籍,多个\chapter{}章节命令
%将TexShop中的“偏好设置”-“编码”,将其设置为“Unicode(UTF-8)”
%使用TexShop编译的时候要用“XeLatex”
%每段开头空两格,不用\\而是编辑时空1行,即可
\usepackage{amsmath} %数学公式宏包
%在行文中,使用$...$插入行内公式,使用\[...\]插入行间公式
% 如果需要对行间公式进行编号,可以使用equation环境:\begin{equaion}...\end{equation}
%网上有:在线LaTeX公式编辑器,方便直观(请百度)
%网上有:在线表格生成器,如:Creat LaTeX tables online(请百度),直观生成表格后再复制LaTeX代码
\usepackage{graphicx} %插入图片宏包
%插入图片方法(将图片宽度缩放至页面宽度的百分之八十):\includegraphics{width = .8 \textwidth}{a.jpg}
\usepackage{pythonhighlight}  %使用Python代码
% 使用方法,将代码录入到:\begin{python}...\end{python}
%缺点,文字太多一行显示不完则自动换行,无进度条,不美观。
\usepackage[colorlinks,linkcolor=red]{hyperref} %使用超链接
%使用方法:\url{http://www.baidu.com} 或者:
% \href{http://www.baidu.com}{BaiDu}
% 抄录命令  \verb |/var/www/html|
\usepackage{listings}  %使用bash
\lstset{               %设置{listings}
 language=bash,                    %设置显示python,也可替换成C++,bash
 frame=single,                       % 设置有边框
% backgroundcolor=\color{black},      % choose the background color
 commentstyle=\color{blue},          % 设置注释颜色
 keywordstyle=\color{blue},          % 设置keyword颜色
 breaklines = true,                  % 代码过长则换行
 showspaces = false,                 % 不显示空格
 showstringspaces=false,             %不显示中间的空格
 escapeinside=``,%逃逸字符(1左面的键),用于显示中文例如在代码中`中文...`
 basicstyle=\ttfamily
}
% 使用方法:\begin{lstlisting}[language={Python}]...python code...\end{lstlisting}
      %行内代码:\verb|bash|
% 缺点:同样存在文字太多一行显示不完则自动换行,无进度条,不美观;且没有彩色显示。
\usepackage{enumerate}
% 黑点无序列表 \begin{itemize} 有序列表 \begin{enumerate} 列表内容用item
% 有序列表 \begin{enumerate}[(1)] item 有序列表 \end{enumerate}
\usepackage{multicol} %用于实现同一页中不同的分栏
% 分两栏显示 \begin{multicols}{2}...\end{multicols} \columnseprule=1pt %实现插入分隔线
\title{基于Debian12.5.0搭建Web服务器\\
       (ThinkPad笔记本)}
\author{编辑:唐宏}
\date{\today}
\begin{document}
\maketitle
%\newpage
\setcounter{tocdepth}{1} %设置目录显示层级
\tableofcontents
%\newpage
%ctexart中定义了五个控制序列来调整行文组织结构,他们分别是:
%\section{}
%\subsection()
%\subsubsection()
%\paragraph()
%\subparagraph{}
\chapter{安装Debian12.5.0系统}
DebianGNU/Linux(简称Debian)是目前世界最大的非商业性Linux发行版之一,是由世界范围1000多名计算机业余爱好者和专业人员在业余时间制作。


安装服务器时选中apache和ssh服务,因为仅让服务器连接电源和网线,其他外设就省了。
\section{基本操作步骤}
\begin{itemize}
\item 启动服务器电脑
\begin{lstlisting}[language={bash}]
# 用户名:
ubuntu
# 密码:
toor
# 在输入密码时,电脑上不会显示
\end{lstlisting}
\item 使用结束,关闭
\begin{lstlisting}[language={bash}]
# 关闭电脑,用root用户
/sbin/shutdown -h now 
\end{lstlisting}
\end{itemize}

\section{自定义IP}
\subsection{设置静态IP}
学校教育内网,linux服务器无法自动获取IP地址,故先在windows系统中通过ipconfig命令查看局域网内IP,再根据查询结果更改ubuntu服务器静态IP地址。更改结果如下:
\begin{lstlisting}[language={bash}]
# 查看IP
ip addr show
# 查看网卡名称
ip a
# sudo vi /etc/network/interfaces
auto enp0s25
iface enp0s25 inet static
address 192.168.1.10
netmask 255.255.255.0
gateway 192.168.1.1
dns-nameservers 114.114.114.114,8.8.8.8
# 应用配置
/etc/init.d/networking restart
\end{lstlisting}


校园内网设置静态IP后也无法连接互联网。
\subsection{设置无线网络}
\begin{lstlisting}[language={bash}]
# sudo vi /etc/network/interfaces
allow-hotplug wls3
iface wls3 inet dhcp
	wpa-ssid Redmi
	wpa-psk 88888888
\end{lstlisting}


网络正常后在局域网内通过Putty软件连接服务器操作
\subsection{ssh远程控制}
\begin{lstlisting}[language={bash}]
# 连接ssh主机
ssh -p 10114 ubuntu@255v99i723.oicp.vip
# 上传文件
scp -P 10114 ./test.tar.gz ubuntu@255v99i723.oicp.vip:/home/ubuntu
\end{lstlisting}


\section{安装网络共享服务}
\subsection{安装Apache2}
Apache HTTP Server(简称Apache)是Apache软件基金会的一个开放源码的网页服务器,Apache(音译为阿帕奇)是世界使用排名第一的Web服务器软件。它可以运行在几乎所有广泛使用的计算机平台上,由于其跨平台和安全性被广泛使用,是最流行的Web服务器端软件之一。它快速、可靠并且可通过简单的API扩充,将Perl/Python等解释器编译到服务器中。


在安装系统时已默认安装。
\begin{lstlisting}[language={bash}]
# 更新包列表
sudo apt update
# 安装apache2
sudo apt install apache2
# 启动Apache服务
sudo systemctl start apache2
\end{lstlisting}


在Ubuntu中,Apache2默认解析目录是 \verb |/var/www/html|,要更改默认解析目录,需要编辑 \verb |/etc/apache2/sites-available/000-default.conf|配置文件。
\begin{lstlisting}[language={bash}]
# 注释掉 DocumentRoot /var/www/html
DocumentRoot /home/ubuntu/html
\end{lstlisting}


编辑 \verb|/etc/apache2/apache2.conf| 文件。修改默认路径 \verb|/var/www/html|为新设置的路径。
\begin{lstlisting}[language={bash}]
<directory /home/ubuntu/html
  Options Indexes FollowSymLinks
  AllowOverride None
  Require all granted
</Directory>
\end{lstlisting}


设置更改后目录的可访问权限。-R是递归设置。
\begin{lstlisting}[language={bash}]
sudo chmod 777 -R /home
sudo systemctl restart apache2
\end{lstlisting}
\subsection{安装vsftpd}
vsftpd 是“very secure FTP daemon”的缩写,安全性是它的一个最大的特点。vsftpd 是一个 UNIX 类操作系统上运行的服务器的名字,它可以运行在诸如 Linux、BSD、Solaris、 HP-UNIX等系统上面,是一个完全免费的、开放源代码的ftp服务器软件,支持很多其他的 FTP 服务器所不支持的特征。比如:非常高的安全性需求、带宽限制、良好的可伸缩性、可创建虚拟用户、支持IPv6、速率高等。
\begin{lstlisting}[language={bash}]
# 更新包列表
sudo apt update
# 安装apache2
sudo apt install vsftpd
# 启动Apache服务
sudo systemctl start vsftpd
\end{lstlisting}


安装好的vsftpd服务器仅只读,要让其可写要修改配置文件。
\begin{lstlisting}[language={bash}]
# sudo vim /etc/vsftpd.conf
write_enable=YES
# 只需去掉语句前的注释符号
\end{lstlisting}

\subsection{安装花生壳软件}
无需依赖公网IP、无需配置路由器,花生壳支持在客户端上添加端口映射,快速将内网服务发布到外网。
\begin{lstlisting}[language={bash}]
wget "https://dl.oray.com/hsk/linux/phddns_5.3.0_i386.deb" -O phddns_5.3.0_i386.deb
apt install ./phddns_5.3.0_amd64.deb
\end{lstlisting}


查询状态并绑定花生壳软件。花生壳内网穿透通过云服务器快速与内网服务器建立连接,同时把内网端口映射到云端,实现各类局域网应用基于域名的互联网访问。
\begin{lstlisting}[language={bash}]
# 查看状态
/bin/phddns status
# 绑定账号
# SN:orayfae6c250f6fb
# Default Password:amdin
# 网站 http://b.oray.com
\end{lstlisting}

\section{文件备份与恢复}
挂载ntfs格式U盘,用于平时查杀U盘病毒,用于备份网站文件等。
\begin{lstlisting}[language={bash}]
# 安装ntfs-3
apt install ntfs-3g
# 查看U盘,结果为/dev/sdb4
/sbin/fdisk -l
# 挂载
/bin/mount -t ntfs-3g /dev/sdb4 /mnt
# 卸载
/bin/umoun /dev/sdb4
\end{lstlisting}


\subsection{wordpress文件备份与恢复}
备份服务器中的wordpress文件夹
\begin{lstlisting}[language={bash}]
#压缩(备份)文件
sudo tar czvf wordpress.tar.gz ./wordpress/
#解压(还原)地图文件
tar xzvf wordpress.tar.gz
\end{lstlisting}

\chapter{安装PHP软件}
想搭建wordpress平台,Veno File Manager VFM网盘平台,都需要php支持。
\section{默认安装php8.2}
默认为8.2,不支持veno File Manager3.4.8上传文件
\begin{lstlisting}[language={bash}]
# 安装
apt install php
apt install php-mysql
# 查看版本
php -v
\end{lstlisting}
可以不默认安装,而直接安装php7.4版本。

\section{多版本php安装}
添加三方源。
\begin{lstlisting}[language={bash}]
apt install software-properties-common ca-certificates lsb-release apt-transport-https
sudo sh -c 'echo "deb https://packages.sury.org/php/ $(lsb_release -sc) main" > /etc/apt/sources.list.d/php.list'
wget -qO - https://packages.sury.org/php/apt.gpg | apt-key add -
\end{lstlisting}


安装其他版本php
\begin{lstlisting}[language={bash}]
apt update
# 安装php7.4
apt install php5.6
apt install php7.4
# 安装mysql支持
apt install php7.4-mysql
# 版本切换, 根据提示选择7.4版本
update-alternatives --config php
\end{lstlisting}

选择7.4版本后,通过\verb|php -v|测出是7.4,但通过info.php测试还是8.2版本,这要更换apache的配置文件。如果直接安装php7.4版本,此步可省略。
\begin{lstlisting}[language={bash}]
# vi /etc/apache2/mods-enabled/php8.2.load
LoadModule php7_module /usr/lib/apache2/modules/libphp7.4.so
\end{lstlisting}


选择php7.4后,就可以安装vfm网盘系统了。只需要上传vfm文件夹,并把uploads文件夹设置权限为777。


\chapter{安装MariaDB服务器}
MariaDB数据库管理系统是MySQL的一个分支,主要由开源社区在维护,采用GPL授权许可 MariaDB的目的是完全兼容MySQL,包括API和命令行,使之能轻松成为MySQL的代替品。MariaDB由MySQL的创始人Michael Widenius主导开发,MariaDB名称来自Michael Widenius的女儿Maria的名字。


安装wordpress需要MySql,debian12.5.0默认无MySql版本。

\section{安装MariaDB}
\begin{lstlisting}[language={bash}]
# 安装MariaDB
apt install mariadb-server
# 查看版本用mysql -v或者:
mariadb -v
# 进入数据库用mysql或者:
mariadb
# 退出数据库用exit;或者:
quit;
\end{lstlisting}


\section{MySql基本操作}


为后期安装博客系统,新建wordpressuser用户、新建wordpress数据库并设置数据库权限。
\begin{lstlisting}[language={bash}]
# 查看数据库
show databases;
# 选择数据库
use mysql;
# 查看表
show tables;
# 查看表内数据
select name,host from user;
# 添加用户
CREATE USER 'wordpressuser'@'localhost' IDENTIFIED BY '123456';
# 新建数据库
CREATE DATABASE wordpress;
# 赋予用户所有权限
GRANT ALL PRIVILEGES ON wordpress.* TO 'wordpressuser'@'localhost';
# 刷新权限使变更生效
FLUSH PRIVILEGES;
# 登陆wordpressuser用户
mysql -u wordpressuser -p
# 输入密码123456即登陆成功
\end{lstlisting}


现在可以安装wordpress博客系统了,要将wp-admin设置权限为777。

\section{MySql备份与恢复}
MySQLdump是MySQL 自带的逻辑备份工具。它的备份原理是通过协议连接到MySQL数据库,将需要备份的数据查询出来,然后把查询出的数据转换成对应的insert语句,当我们需要还原这些数据时,只要执行这些insert语句,即可将对应的数据还原。


MySQLdump可以用来做完全备份和部分备份,支持InnoDB存储引擎的热备功能,以及MyISAM存储引擎的温备功能。让我们一起来看看MySQL怎么备份吧。


备份wordpress数据库。
\begin{lstlisting}[language={bash}]
# 备份wordpress数据库,返回到shell窗口
sudo mysqldump -u wordpressuser -p 123456 -P 3306 wordpress > wp.sql
# 简化备份,输入密码
mysqldump -u wordpressuser -p wordpress > wp.sql
# 还原wordpress数据库。
mysql -u wordpressuser -p wordpress < wp.sql
\end{lstlisting}


\chapter{安装Docker}
Docker 是一个开源的应用容器引擎,让开发者可以打包他们的应用以及依赖包到一个可移植的镜像中,然后发布到任何流行的 Linux或Windows操作系统的机器上,也可以实现虚拟化。容器是完全使用沙箱机制,相互之间不会有任何接口。 


想通过docker搭建小皮面板,再通过docker搭建LAMP服务器,实现同一台物理机中有两个虚拟网站,小皮面板搭建的wordpress能通过手机APP访问,便于手机访问,便于备份,便于移植。


基于debian12.5.0安装Docker,也在kali系统中成功安装docker,方法类似但并不完全一样,请自行百度。安装困难,要先在虚拟机中测试成功后再实施。


\begin{lstlisting}[language={bash}]
apt install wordpress
\end{lstlisting}

\end{document}